\documentclass[twocolumn,aps,prd,nofootinbib]{revtex4-2}

\usepackage{amsmath}
\usepackage{amsfonts}
\usepackage{amssymb}
\usepackage{array}
\usepackage{graphicx} % Required for inserting images
\graphicspath{{figs/}{build/figs/}}

\begin{document}

\csname title\endcsname{The X-Theta Framework: Loop-Dependent Phase Transport via Fiber-Bundle Connections and a 4D St\"uckelberg EFT}
\author{Divyang Panchasara}
\affiliation{Gemini Research Institute}
\date{\today}

\begin{abstract}
We present the X--$\theta$ framework: an effective geometric extension in which an internal compact phase coordinate $\theta$ is transported by a U(1) connection over spacetime. The central observable is a loop-phase response, quantified by the transport phase
\begin{equation}
\Delta\theta_{\mathrm{loop}}(\gamma)=-\kappa\oint_\gamma A\quad(\mathrm{mod}\,2\pi),
\end{equation}
which depends primarily on loop geometry (flux/area/winding) rather than traversal time. We formulate the gauge-invariant bundle/St\"uckelberg structure, relate it to Wilson loops and geometric phase, and outline how standard covariant derivatives arise from dimensional reduction on an $S^1$ fiber. A minimal 4D completion includes an explicit St\"uckelberg kinetic term $\tfrac{f^2}{2}(\partial_\mu\theta+\kappa A_\mu)^2$, which provides dynamics for the phase field and induces a mediator mass scale. We propose a three-tier validation suite (simulation, tabletop analogs, and open-data stress tests) with explicit falsifiers for numerical artifacts and speed-dominated drift. Consistency constraints are discussed, including Yukawa-suppressed long-range deviations to preserve agreement with solar-system gravity. We emphasize that Bell-violating correlations, if reproduced, require an explicit relaxation of at least one Bell assumption; the framework provides language for such hypotheses but no automatic loophole.
\end{abstract}

\maketitle

\section{Introduction}

\subsection{Motivation: from ``local forces'' to ``loop-phase response''}
% Content from work.md
Standard classical dynamics is local: the state at time ($t$) determines the evolution at ($t+dt$) through force laws. Quantum mechanics adds phase, but typically treats it as either (i) a gauge artifact or (ii) a wavefunction property without an explicit geometric ``mechanism.''

The X--$\theta$ framework explores a minimal geometric extension: add a compact internal coordinate ($\theta$) and allow its evolution to depend on closed loops in X through a connection. This yields an operationally testable claim:

\begin{quote}
        	\textbf{Claim (Loop-phase invariance):} for closed loops ($\gamma$), the net transport phase ($\Delta\theta_{\mathrm{loop}}(\gamma)$) depends primarily on loop geometry (area/winding/flux), and only weakly on traversal details (speed profile), up to controlled noise.
\end{quote}

\subsection{Relation to known physics}
% Content from work.md
X--$\theta$ sits in the intersection of three established ideas:
\begin{enumerate}
    \item \textbf{Gauge connections and Wilson loops:} loop integrals ($\oint A\cdot dx$) are physical through gauge-invariant Wilson loops / geometric phase \cite{Wilson1974,Nakahara2003,AharonovBohm1959}.
    \item \textbf{Geometric phases:} Berry-type phases depend on closed paths in parameter space \cite{Berry1984}.
    \item \textbf{Kaluza--Klein style decompositions:} compact directions can appear as gauge structure \cite{Kaluza1921, Klein1926}, but X--$\theta$ treats $\theta$ as an \textbf{effective} compact internal degree of freedom, not necessarily Planck-scale.
\end{enumerate}
The framework is not presented as ``proof of new physics'' but as a \textbf{working EFT} with \textbf{kill tests}.

\subsection{Bridge from our original checklist (what the new formalism actually solves)}
% Content from work.md (table)
Moving to a principal-bundle / St\"uckelberg framing cleanly separates what is \textbf{definition-level} (gauge-invariant and already rigorous) from what is \textbf{model-level} (extra dynamical assumptions that must be tested).

\begin{table*}[!htbp]
\caption{Bridge from our original checklist}
\begin{ruledtabular}
\begin{tabular}{r l >{\raggedright\arraybackslash}p{0.54\textwidth} >{\raggedright\arraybackslash}p{0.16\textwidth}}
                Item & Topic & What the bundle/St\"uckelberg math gives immediately & Status in this draft \\
\colrule
1 & AB / loop phase & For a base loop $\gamma\subset X$, the physical fiber shift is defined by parallel transport, giving $\Delta\theta_{\mathrm{loop}}(\gamma)=-\kappa\oint_\gamma A$ (mod $2\pi$). By Stokes, $\Delta\theta_{\mathrm{loop}}(\gamma)=-\kappa\iint_S F$ for any spanning surface $\partial S=\gamma$. AB regime: $F=0$ along the path but nontrivial enclosed flux/topology. & \textbf{Included + Tier-1 tests} \\
2 & Mercury / gravity & Provides a clean place to add a short-range scalar/mediator (e.g., from a St\"uckelberg mode) and compute weak-field corrections; also forces explicit observational constraints. & \textbf{Future work / constraints} \\
3 & Schr\"odinger recovery & Makes ``minimal coupling'' geometric: covariant derivatives appear once $\omega=d\theta+\kappa A$ is the invariant phase 1-form; a full derivation depends on specifying a matter action. & \textbf{Partial (mapping only)} \\
4--5 & Bell / EPR & Provides language for hypotheses (shared fiber phase / joint connection), but does not automatically evade Bell. Any completion must specify a probability map and clearly state which Bell assumption is relaxed. & \textbf{Hypothesis only (flagged)} \\
6 & Singularity & Bundle language can host stabilization mechanisms (fiber freezing, higher-curvature terms), but this is a dynamical claim requiring an explicit high-curvature completion. & \textbf{Outlook / speculative} \\
7 & Forces & Standard gauge logic: curvature $F$ encodes force-like effects. ``Unification'' is a separate claim requiring explicit EFT couplings plus tests. & \textbf{Framed as EFT} \\
\end{tabular}
\end{ruledtabular}
\label{tab:checklist}
\end{table*}

\section{Geometric skeleton: bundle, connection, transport}

\subsection{Principal-bundle viewpoint (reviewer-proof wording)}
% Content from work.md
Instead of treating $\theta$ as a rigid fifth coordinate, treat it as a \textbf{local fiber coordinate} on a principal U(1) bundle $\pi:P\to X$. The connection is the physical object; $\theta$ and $A_\mu$ are individually gauge-dependent, but the St\"uckelberg combination is gauge-invariant.

Operationally, define the invariant 1-form $\omega \equiv d\theta + \kappa A_\mu(x) dx^\mu$. Under a local gauge parameter $\Lambda(x)$:
\begin{itemize}
    \item $A_\mu \to A_\mu - (1/\kappa) \partial_\mu\Lambda$
    \item $\theta \to \theta + \Lambda$
\end{itemize}
so $\omega$ is unchanged.

\paragraph{Important point (fixing a common ambiguity).}
The 1-form $\omega$ lives on the \emph{total space} (it requires both $x(t)$ and $\theta(t)$). A loop $\gamma\subset X$ in the base does \emph{not} by itself define a curve in $P$ (or $Q$) unless we specify a lift. The physically meaningful ``net shift in $\theta$'' is therefore defined via \textbf{horizontal lift / parallel transport}.

Concretely: given a base path $x(t)$, define its horizontal lift $\Gamma(t)=(x(t),\theta(t))$ by
\begin{equation}
\omega(\dot{\Gamma})=0\quad\Longleftrightarrow\quad \dot\theta(t)=-\kappa\,A_\mu(x(t))\,\dot x^\mu(t).
\end{equation}
For a base loop $\gamma$ with $x(T)=x(0)$, the transported fiber angle shifts by
\begin{equation}
\Delta\theta_{\mathrm{loop}}(\gamma)\equiv \theta(T)-\theta(0)= -\kappa\oint_\gamma A\quad(\mathrm{mod}\,2\pi).
\end{equation}
This is the loop-phase response observable. (Along the horizontal lift one has $\oint_\Gamma \omega=0$ by definition; the measured quantity is the endpoint mismatch in $\theta$ after transport.)

This wording matches what reviewers expect from geometric-phase physics: the observable is a Wilson-loop-like quantity, not a coordinate \cite{Wilson1974,Nakahara2003}.

\subsection{Gauge invariance}
Define a local gauge transformation by a scalar ($\Lambda(x)$):
\begin{equation}
\theta \to \theta' = \theta + \Lambda(x),
\qquad
 A_\mu \to A_\mu' = A_\mu - \frac{1}{\kappa}\,\partial_\mu \Lambda.
\end{equation}
Then
\begin{align}
\omega' &= d\theta' + \kappa A_\mu' dx^\mu \\
&= (d\theta + d\Lambda) + \kappa\left(A_\mu - \frac{1}{\kappa}\partial_\mu\Lambda\right)dx^\mu \\
&= d\theta + \kappa A_\mu dx^\mu \\
&= \omega.
\end{align}
So ($\omega$) is gauge invariant.

\subsection{Loop phase as observable}
For a closed loop ($\gamma\subset X$):
\begin{equation}
\Delta\theta_{\mathrm{loop}}(\gamma)
\equiv -\kappa \oint_\gamma A_\mu dx^\mu
\quad (\mathrm{mod}\,2\pi).
\end{equation}
Equivalently, define the Wilson loop element in U(1):
\begin{equation}
W(\gamma) = \exp\Big(i\kappa\oint_\gamma A_\mu dx^\mu\Big) = e^{-i\Delta\theta_{\mathrm{loop}}(\gamma)}.
\end{equation}

\subsection{Curvature and Stokes' theorem}
The curvature two-form is
\begin{equation}
F \equiv dA = \frac{1}{2}F_{\mu\nu}dx^\mu\wedge dx^\nu,
\qquad
 F_{\mu\nu}=\partial_\mu A_\nu-\partial_\nu A_\mu.
\end{equation}
For a surface ($S$) with boundary ($\partial S = \gamma$), Stokes gives
\begin{equation}
\oint_\gamma A = \int_S dA = \int_S F.
\end{equation}
Hence
\begin{equation}
\Delta\theta_{\mathrm{loop}}(\gamma) = -\kappa\int_S F.
\end{equation}
This is the precise statement of ``geometry dominates traversal'': the loop phase depends on the \textbf{flux of curvature} through the loop.

\subsection{Units \& scales (minimal sanity)}
I take $\theta$ to be dimensionless (an angle), so $d\theta$ is dimensionless. Therefore $\kappa A_\mu dx^\mu$ must also be dimensionless, implying $[\kappa A_\mu]=\mathrm{length}^{-1}$ (and for $\mu=0$ equivalently time$^{-1}$).
Equivalently, one may treat $A_\mu$ as having inverse-length units and $\kappa$ dimensionless, or treat $A_\mu$ as carrying conventional field units and let $\kappa$ carry the compensating dimensions; either convention is acceptable provided $\kappa\oint A\cdot dx$ is dimensionless.

In the hydrodynamic proxy $\mathbf A=\alpha_{\mathrm{eff}}\mathbf u$ with $[\mathbf u]=\mathrm{length}/\mathrm{time}$, the coefficient must satisfy $[\alpha_{\mathrm{eff}}]=[\mathbf A]/[\mathbf u]$ so that $\kappa\oint \mathbf A\cdot d\boldsymbol{\ell}$ is dimensionless. This bookkeeping is not cosmetic: it pins down what ``small coupling'' means and prevents scale-driven contradictions between lab-scale phases and solar-system constraints. We treat $\mathbf A=\alpha_{\rm eff}\mathbf u$ as a \emph{modeling choice}; only the loop functional $\oint_\gamma \mathbf A\cdot d\boldsymbol{\ell}$ (equivalently $\int_S \nabla\times\mathbf A\cdot d\mathbf S$) is claimed observable.

\section{Emergence of Quantum Dynamics from the Fiber}\label{sec:emergence}

The geometric structure of the $Q = X \times S^1$ space provides a natural origin for the minimal coupling prescription of quantum mechanics. By considering a 5D field on this extended space and decomposing it into fiber harmonics, we can show that the resulting 4D effective theory naturally contains the covariant derivative, leading to the Klein-Gordon and, in the non-relativistic limit, the Schr\"odinger equation.

\subsection{A minimal action (EFT starting point)}
A field-theoretic presentation begins by specifying an action that couples matter to the connection. A minimal gauge-invariant choice (sufficient for an EFT-level discussion) is
\begin{equation}
S = \int d^4x \, \sqrt{-g} \left[ -\frac{1}{4}F_{\mu\nu}F^{\mu\nu} + (D_\mu\Phi)^\ast (D^\mu\Phi) - V(\Phi) + \frac{f^2}{2}\, (\partial_\mu\Theta + \kappa A_\mu)(\partial^\mu\Theta + \kappa A^\mu) \right],
\end{equation}
where $\Phi$ is a complex scalar matter field, $D_\mu\equiv \nabla_\mu-i\kappa A_\mu$, and $V(\Phi)$ encodes self-interactions and any symmetry-breaking scale. The field $\Theta(x)$ is a 4D St\"uckelberg field with decay constant $f$; the gauge-invariant combination $\partial_\mu\Theta+\kappa A_\mu$ provides explicit 4D dynamics and generates a vector mass $m_A=\kappa f$ in unitary gauge.

\subsection{The Fiber Metric and 5D Klein-Gordon Equation}

We begin by defining the metric on the total space $Q$. We extend the 4D spacetime metric $g_{\mu\nu}$ with a fiber component, where the connection $A_\mu$ is incorporated into the metric itself:
\begin{equation}
ds^2 = G_{AB}\,\mathrm{d}q^A\,\mathrm{d}q^B = g_{\mu\nu}\,\mathrm{d}x^\mu\,\mathrm{d}x^\nu + R^2\big(\mathrm{d}\theta + \kappa A_\mu\,\mathrm{d}x^\mu\big)^2
\end{equation}
Here, $R$ is the radius of the $S^1$ fiber. A free scalar field $\Phi(x, \theta)$ on this 5D space obeys the Klein-Gordon equation:
\begin{equation}
\frac{1}{\sqrt{-G}}\partial_A(\sqrt{-G}G^{AB}\partial_B\Phi) - M^2\Phi = 0
\end{equation}
where $M$ is the 5D mass of the field.

\subsection{Harmonic Expansion and Dimensional Reduction}

Since the fiber is a compact circle, we can expand $\Phi(x, \theta)$ in a Fourier series of fiber harmonics:
\begin{equation}
\Phi(x, \theta) = \sum_{n=-\infty}^{\infty} \Phi_n(x) e^{in\theta}
\end{equation}
where $\Phi_n(x)$ are the 4D field components for each mode $n$.

Substituting this expansion into the Klein-Gordon equation and integrating over $\theta$, we perform a dimensional reduction. The derivatives with respect to $\theta$ become algebraic terms involving $n$. The crucial step is the appearance of the covariant derivative. The 4D operator that acts on the $\phi_n(x)$ modes is precisely the minimally coupled d'Alembertian:
\begin{equation}
\left( D_\mu D^\mu + m_n^2 \right) \Phi_n(x) = 0
\end{equation}
where the covariant derivative is given by:
\begin{equation}
D_\mu = \nabla_\mu - i\,n\kappa\, A_\mu
\end{equation}
which is the standard KK result: schematically $\partial_\mu\to \partial_\mu+\kappa A_\mu\partial_\theta$, and for a mode $e^{in\theta}$ one has $\partial_\theta\to in$. Thus the effective 4D charge of the $n$-th mode is $q_n\equiv n\kappa$ (up to conventions for whether $\theta$ is dimensionless and how $R$ is absorbed). The KK mass tower is
\begin{equation}
m_n^2 = M^2 + \frac{n^2}{R^2}.
\end{equation}

\subsection{Non-relativistic Limit: The Schr\"odinger Equation}

In the non-relativistic limit, we consider the $n=1$ mode and write the field as $\Phi_1(x) = e^{-imt} \psi(x,t) / \sqrt{2m}$. Taking the low-energy limit of the Klein-Gordon equation, we recover the familiar Schr\"odinger equation for a particle of charge $q$ in an electromagnetic potential $A_\mu$:
\begin{equation}
i\hbar \frac{\partial\psi}{\partial t} = \left[ \frac{1}{2m}(-i\hbar\nabla - q\mathbf{A})^2 + qA_0 \right]\psi
\end{equation}
where $A_0$ is the scalar potential.

This derivation demonstrates that the gauge coupling and phase structure of quantum mechanics are not ad-hoc additions but emerge naturally from the geometry of the $S^1$ fiber bundle. It provides a geometric origin for the principle of minimal coupling.


\section{Bell correlations: scope and constraints}\label{sec:bell}

The X--$\theta$ framework provides a compact geometric language for \emph{non-separable phase structure} on enlarged configuration spaces (e.g., a joint two-particle space $Q_2$). This should not be read as an automatic loophole to Bell's theorem. Any completion that reproduces Bell-violating correlations must explicitly state which Bell assumption is relaxed (typically local causality / parameter independence) and must separately enforce operational no-signalling.

\paragraph{Explicit no-signalling constraint.}
Any proposed model must satisfy
\begin{equation}
P(M_A\mid a,b)=P(M_A\mid a),\qquad P(M_B\mid a,b)=P(M_B\mid b),
\end{equation}
in addition to whatever non-local dependence is introduced at the hidden-variable or joint-connection level.

Details (including a toy generative model and a reference plot of the singlet curve) are moved to Appendix~\ref{app:bell-toy}.


\section{Phenomenology II: hydrodynamic ansatz (macroscopic analog)}

\subsection{Map from fluid velocity to connection}
% Content from work.md
Let ($\mathbf{u}(x,t)$) be an incompressible velocity field (($\nabla\cdot\mathbf{u}=0$)). Postulate
\begin{equation}
\mathbf{A}(x,t) = \alpha_{\mathrm{eff}}\,\mathbf{u}(x,t).
\end{equation}
Then curvature becomes proportional to vorticity:
\begin{equation}
\nabla\times\mathbf{A} = \alpha_{\mathrm{eff}}\,(\nabla\times\mathbf{u}) = \alpha_{\mathrm{eff}}\,\boldsymbol{\varpi}.
\end{equation}

\subsection{Loop phase equals circulation / vorticity flux}
% Content from work.md
For a material loop ($\gamma$):
\begin{equation}
\Delta\theta_{\mathrm{loop}}(\gamma) = -\kappa\oint_\gamma \mathbf{A}\cdot d\mathbf{\ell}
= -\kappa\alpha_{\mathrm{eff}}\oint_\gamma \mathbf{u}\cdot d\mathbf{\ell}
= -\kappa\alpha_{\mathrm{eff}}\int_S \boldsymbol{\varpi}\cdot d\mathbf{S}.
\end{equation}
Thus \textbf{winding/area/vorticity flux} statistics can be used as topological predictors of ($\Delta\theta$). 

\subsection{Worked scaling example (solid-body rotation; $\alpha_{\mathrm{eff}}$ sanity)}
To make the scaling concrete, consider a 2D solid-body rotation field $\mathbf u(\mathbf r)=\boldsymbol{\Omega}\times\mathbf r$ (constant angular velocity). Then the vorticity is uniform: $\boldsymbol{\varpi}=\nabla\times\mathbf u=2\boldsymbol{\Omega}$. For a planar loop of signed area $A$ with normal aligned with $\boldsymbol{\Omega}$,
\begin{equation}
\oint_\gamma \mathbf u\cdot d\boldsymbol{\ell}=\int_S \boldsymbol{\varpi}\cdot d\mathbf S = 2\,\Omega\,A,
\end{equation}
so the loop-phase response becomes
\begin{equation}
\Delta\theta_{\mathrm{loop}}\approx -2\,\kappa\,\alpha_{\mathrm{eff}}\,\Omega\,A\quad(\mathrm{mod}\,2\pi).
\end{equation}
This example makes three intended points: (i) the dependence is geometric ($A$) rather than traversal-time dominated, (ii) the sign is set by orientation, and (iii) $\kappa\alpha_{\mathrm{eff}}$ is fixed by a single amplitude calibration in any hydrodynamic proxy implementation.

\section{Validation suite (three tiers): simulation + tabletop + open data}

\noindent\textbf{Tier meaning.} Tier 1 is \textbf{pipeline validation + numerical falsifiers} (time-warp invariance, step-size stability, and pure-gauge controls). Tier 2/3 are the first places the framework becomes \emph{physically} testable once an independently measured dial variable is implemented.

\subsection{Shared falsification principle}
% Content from work.md
For each tier we test whether the measured/inferred loop-phase statistic is:
\begin{itemize}
    \item \textbf{Geometry-dominated:} correlates with area/winding/flux
    \item \textbf{Speed-robust:} weakly dependent on traversal time or speed profile
    \item \textbf{Numerically stable:} robust to sampling rate and integration step size
\end{itemize}
We enforce a ``fit-once / predict-many'' rule: ($\kappa$) is fit on a calibration subset then held fixed.

\subsection{Tier 1 --- pipeline validation (AB-like loop functional)}
% Content from work.md
\subsubsection*{T1-Sim: clean 2D simulation}
\begin{enumerate}
    \item Choose $\mathbf{A}(x,y)=\tfrac{B}{2}(-y,\,x)$.
    \item Generate loop families:
    \begin{itemize}
        \item same geometry, different speed profiles
        \item same area, different shapes
        \item same duration, different area
    \end{itemize}
    \item Integrate ($\Delta\theta_{\mathrm{loop}} = -\kappa\oint \mathbf{A}\cdot d\mathbf{x}$) by discrete line integral.
    \item Plot:
    \begin{itemize}
        \item ($\Delta\theta$) vs ($A_{\mathrm{signed}}$) (collapse)
        \item ($\Delta\theta$) vs duration (should not collapse)
    \end{itemize}
\end{enumerate}

    \noindent\textbf{Pipeline falsifier:} if duration explains ($\Delta\theta$) better than area in a static $\mathbf{A}(\mathbf{x})$, reject the estimator/pipeline (this indicates time-warp sensitivity or numerical artifacts, not new physics).

\begin{figure}[htbp]
    \centering
    \includegraphics[width=0.95\linewidth]{cross_hall_drift_baseline.png}
    \caption{Representative baseline output from the simulation pipeline (cross-Hall drift observable).}\label{fig:cross-hall-baseline}
\end{figure}

\subsubsection*{T1-Tabletop: phone-loop (estimator sanity check)}
\begin{itemize}
    \item Record trajectory (indoor VO or IMU fusion).
    \item Reconstruct ($x(t)$), compute ($A_{\mathrm{signed}}$).
    \item Compute the loop functional ($\Delta\theta_{\mathrm{loop}}=-\kappa\oint \mathbf{A}\cdot d\mathbf{x}$) and verify invariance under time warps / resampling.
\end{itemize}
    \noindent\textbf{Control:} same path, time-warped resampling $\to$ same ($\Delta\theta_{\mathrm{loop}}$).
    \noindent\textbf{Scope note:} this validates the estimator and robustness pipeline, not a direct measurement of a physical $\theta$ degree of freedom.

\subsubsection*{T1-Tabletop: physical dial (proposed falsifiable implementation)}
To avoid a closed loop of definitions, the tabletop test must measure a dial variable independently. One minimal option is to realize $\theta$ as an instrumented rotor/oscillator phase (encoder or phase meter) and test whether its measured net shift correlates with $\oint A\cdot dx$ under controlled loop families. In this draft, I treat this as a proposed implementation detail rather than a completed experimental claim.

\subsubsection*{T1-Open data: loop invariance with ground truth trajectories}
\begin{itemize}
    \item Extract loop segments from public VI datasets.
    \item Compute ($A_{\mathrm{signed}}$).
    \item Stress-test with time-warps.
\end{itemize}
Goal: demonstrate estimator invariance under time warps and resampling.

\subsection{Tier 2 --- loop-phase statistics under turbulence}
% Content from work.md
\subsubsection*{T2-Sim: synthetic turbulence + tracer loop-phase response}
\begin{enumerate}
    \item Build incompressible ($\mathbf{u}(x,t)$) via Fourier modes.
    \item Advect tracers: ($\dot x = u(x,t)$).
    \item Define loop statistics in sliding windows:
    \begin{itemize}
        \item signed area ($A_w$)
        \item winding number ($n_w$) about vortices
    \end{itemize}
    \item Set ($\mathbf{A}=\alpha\mathbf{u}$) and integrate ($\Delta\theta$).
\end{enumerate}
\textbf{Success metric:}
\begin{equation}
\mathbb{E}[\Delta\theta\mid n_w=n] \approx n\,C\quad\text{stable as noise increases.}
\end{equation}
\noindent\textbf{Negative control:} pure gauge ($\mathbf{A}=\nabla\chi\Rightarrow \Delta\theta\approx 0$).

\subsubsection*{T2-Tabletop: shallow tank + tracer tracking}
\begin{itemize}
    \item Overhead camera; track dye blobs.
    \item Compute winding/area stats.
    \item Compare predicted ($\Delta\theta$) statistics to dial evolution.
\end{itemize}

\subsubsection*{T2-Open data: turbulent benchmark injection}
\begin{itemize}
    \item Load velocity fields from a public turbulence dataset.
    \item Advect tracers and compute loop-phase statistics.
\end{itemize}
Deliverables:
\begin{itemize}
    \item ($\mathbb{E}[\Delta\theta]$) vs winding number with confidence bands
    \item slope stability vs step size and added noise
\end{itemize}

\subsection{Tier 3 --- spectral sideband fingerprint}
% Content from work.md
\subsubsection*{T3-Sim: controlled phase modulation}
Let ($\\theta(t)=\\omega_\theta t$) and
($s(t)=\\cos(\\omega_0 t+\\beta\\theta(t))$).
Then spectral lines appear at
($\\omega_0\pm m\\omega_\theta$).

In the sinusoidal PM case, sideband amplitudes follow Bessel scaling:
\begin{equation}
\cos(\\omega_0 t + \\beta\sin\\omega_\theta t)
= \sum_{m=-\infty}^{\\infty} J_m(\\beta)\,\cos((\\omega_0+m\\omega_\theta)t).
\end{equation}
So a clean falsifier is: sideband power tracks ($J_m(\\beta)$) under controlled ($\\beta$) sweeps.

\subsubsection*{T3-Tabletop: vibration or audio carrier}
\begin{itemize}
    \item Use motor vibration or audio tone as carrier.
    \item Fit PM model vs baseline AM/FM.
    \item Confirm identifiability and parameter stability.
\end{itemize}

\subsubsection*{T3-Open data: hostile benchmark for inference}
\begin{itemize}
    \item Apply sideband detection and model comparison to open vibration datasets.
    \item Goal is not ``discovering new physics,'' but stress-testing inference and identifiability.
\end{itemize}

\section{Constraints, Limits, and Consistency Checks}

\subsection{Decoupling limits}
% Content from work.md
\begin{itemize}
    \item ($\kappa \to 0$): the dial decouples (no loop-phase response).
    \item Flat curvature ($F\to 0$): the loop phase vanishes for contractible loops.
\end{itemize}

\subsection{Micro-Macro Scale Separation}
% Content for Micro-Macro Scale Separation
The X--$\theta$ framework distinguishes between microscopic and macroscopic manifestations of the $\theta$ phase:
\begin{itemize}
    \item \textbf{Microscopic:} $\theta$ phases are associated with coherent Quantum Interference and Memory effects. These are expected to be observable in systems where quantum coherence is maintained, such as in Aharonov-Bohm type experiments or other geometric phase phenomena.
    \item \textbf{Macroscopic:} In the macroscopic (incoherent vacuum) regime, there is no observable force on planets, and Standard General Relativity (GR) applies. One concrete way to enforce this is a \emph{short-range mediator}: if the effective $\theta$-sector mediator has mass $m_\theta$ (range $\lambda_\theta\equiv 1/m_\theta$), then any static fifth-force-like modification is Yukawa-suppressed as $\propto e^{-r/\lambda_\theta}$ and is negligible for $r\gg\lambda_\theta$.
\end{itemize}
This distinction is crucial for maintaining consistency with observed physics across different scales.

In the standard parameterization used for solar-system constraints, this suppression can be written as an effective Yukawa correction to the Newtonian potential,
\begin{equation}
V_{\text{eff}}(r) = -\frac{GM}{r} \left( 1 + \alpha e^{-r/\lambda_\theta} \right),
\end{equation}
where $\alpha$ is a dimensionless strength and $\lambda_\theta$ sets the range.

\noindent\textbf{Single coupling choice (explicit).} For concreteness, take the minimal St\"uckelberg completion to be a massive $U(1)$ vector $A_\mu$ with action
\begin{equation}
S=\int d^4x\,\sqrt{-g}\left[-\frac{1}{4}F_{\mu\nu}F^{\mu\nu}+\frac{f^2}{2}(\partial_\mu\theta+\kappa A_\mu)^2 + g_\theta\,A_\mu J^\mu\right],
\end{equation}
where $m_\theta=g_\theta f$ sets $\lambda_\theta\equiv 1/m_\theta$. As a concrete low-energy coupling to ordinary matter, take $J^\mu$ to be the (approximately conserved) nonrelativistic mass current, $J^\mu\approx (\rho,\,\rho\mathbf v)$. In the static limit, exchange of $A_\mu$ yields a Yukawa-suppressed long-range correction, which can be mapped onto the standard $(\alpha,\lambda_\theta)$ parameterization used for solar-system bounds.

\section{Discussion}
% Content from work.md
The X--$\theta$ framework is a structured way to ask a specific question: \textbf{can loop-dependent phase transport act as an effective, measurable channel in physical systems, beyond the contexts where geometric phase is already known?}

The validation suite is intentionally designed to fail fast under:
\begin{itemize}
    \item duration-dominated drift,
    \item numerical artifacts,
    \item uncontrolled gauge choices,
    \item and overfitting ($\kappa$) or ($\varepsilon$).
\end{itemize}
If the framework survives Tier 2 under harsh robustness tests, it becomes scientifically interesting irrespective of ultimate interpretation.

\section{Discussion and Outlook}\label{sec:conclusion}

This paper frames X--$\theta$ as an effective geometric model with a single operational target: loop-dependent phase transport. The core contribution is not a claim of discovered new physics, but a consistent bundle/St\"uckelberg formulation plus a validation suite designed to fail fast on numerical artifacts and time-dominated drift.

\subsection{Speculative outlook: high-curvature regime}
Several motivations that initially prompted this project (e.g., behavior near strong gravity or high curvature) require an explicit dynamical completion beyond the kinematic transport law. In particular, any claim about gravitational collapse or singularity structure depends on the detailed stress-energy content and field equations of a specified completion, and is therefore deferred to future work.

\subsection{Speculative outlook: coupling structure}
Similarly, broader "unification" language is inappropriate without explicit couplings and quantitative constraints. Here we restrict to a minimal $U(1)$ St\"uckelberg EFT with a single matter current coupling and emphasize that consistency with existing bounds is a primary requirement, not an afterthought.

\subsection{Conclusion}
The framework is scientifically useful if it can (i) define loop-phase observables cleanly, (ii) produce robust estimators whose failures are diagnosable, and (iii) motivate concrete tests where geometry (area/flux/winding) rather than traversal time dominates. The next step is to execute Tier-1 falsifiers, and only then escalate to Tier-2/Tier-3 settings with independently measured dial variables.


\appendix
\section{Toy joint-connection model for Bell-style correlations}\label{app:bell-toy}

This appendix collects material that is \,\emph{not} needed for the mainline claims of this paper (loop-phase transport and its falsifiers), but which may be useful as a bookkeeping example.

\subsection{Reference result: the quantum singlet correlation}
For spin-$\tfrac{1}{2}$ particles prepared in the singlet state $\lvert\psi^-\rangle$, the quantum prediction for the correlation of outcomes along analyzer directions $\hat a$ and $\hat b$ is
\begin{equation}
E(\hat a,\hat b)=\langle \psi^-\rvert\,\sigma_{\hat a}\otimes\sigma_{\hat b}\,\lvert\psi^-\rangle = -\hat a\cdot\hat b.
\end{equation}
For coplanar analyzer angles $a,b$, this reduces to $E(a,b)=-\cos(a-b)$.

\begin{figure}[htbp]
	\centering
	\includegraphics[width=0.95\linewidth]{bell_correlation.png}
	\caption{Reference correlation curve: the quantum singlet prediction $E(a,b)=-\cos(a-b)$ versus a simple deterministic local sign model.}\label{fig:bell-correlation}
\end{figure}

\subsection{Non-factorization on joint configuration space}
In the X--$\theta$ language, a two-particle system is described on the joint configuration space $Q_2 = (X_1 \times S^1_1) \times (X_2 \times S^1_2)$. A structural possibility is that the relevant connection on this joint space does not decompose into a sum of connections on the individual particle spaces. This \emph{non-factorizability} is a compact way to represent non-separable phase structure, but it is not by itself a derivation of entanglement.

\subsection{Toy generative model (explicitly non-local)}
To go beyond plotting the quantum prediction, one must specify a generative model for measurement outcomes. The following is a \emph{toy} model included only to make the dependence structure explicit.

Let the measurement outcome for particle A be
\begin{equation}
M_A=\mathrm{sign}\!\bigl(\cos(a-\lambda)\bigr),\qquad \lambda\sim \mathrm{Unif}[0,2\pi),
\end{equation}
and for particle B
\begin{equation}
M_B=\mathrm{sign}\!\bigl(\cos(b-\lambda-\pi-c\,(a-b))\bigr),
\end{equation}
where $c$ is a coupling parameter. Because $M_B$ depends on $(a,b)$ through the argument shift, this violates local causality (parameter independence).

\subsection{Explicit no-signalling constraint}
Any proposed completion that relaxes local causality must still enforce operational no-signalling. In standard notation:
\begin{equation}
P(M_A\mid a,b)=P(M_A\mid a),\qquad P(M_B\mid a,b)=P(M_B\mid b).
\end{equation}
In the toy model above, the marginal of $M_A$ is $P(M_A=\pm 1)=\tfrac{1}{2}$ for all $a,b$ because $(a-\lambda)$ is uniformly distributed modulo $2\pi$; similarly for $M_B$ because its argument is also a uniformly shifted angle. This property must be checked (analytically or numerically) in any more elaborate model.

\subsection{Warning: no free lunch}
If a simulation returns $-\cos(\Delta)$ by definition, it is not a model but a plotter. A legitimate simulation must specify a generative rule for outcomes, state which Bell assumption is being relaxed, and separately verify no-signalling.


\bibliographystyle{apsrev4-2}
\bibliography{refs} % This will be references.bib

\end{document}