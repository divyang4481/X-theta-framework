\section{Discussion and Outlook}\label{sec:conclusion}

This paper frames X--$\theta$ as an effective geometric model with a single operational target: loop-dependent phase transport. The core contribution is not a claim of discovered new physics, but a consistent bundle/St\"uckelberg formulation plus a validation suite designed to fail fast on numerical artifacts and time-dominated drift.

\subsection{Speculative outlook: high-curvature regime}
Several motivations that initially prompted this project (e.g., behavior near strong gravity or high curvature) require an explicit dynamical completion beyond the kinematic transport law. In particular, any claim about gravitational collapse or singularity structure depends on the detailed stress-energy content and field equations of a specified completion, and is therefore deferred to future work.

\subsection{Speculative outlook: coupling structure}
Similarly, broader "unification" language is inappropriate without explicit couplings and quantitative constraints. Here we restrict to a minimal $U(1)$ St\"uckelberg EFT with a single matter current coupling and emphasize that consistency with existing bounds is a primary requirement, not an afterthought.

\subsection{Conclusion}
The framework is scientifically useful if it can (i) define loop-phase observables cleanly, (ii) produce robust estimators whose failures are diagnosable, and (iii) motivate concrete tests where geometry (area/flux/winding) rather than traversal time dominates. The next step is to execute Tier-1 falsifiers, and only then escalate to Tier-2/Tier-3 settings with independently measured dial variables.
