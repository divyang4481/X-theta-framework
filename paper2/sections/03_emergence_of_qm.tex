\section{Emergence of Quantum Dynamics from the Fiber}\label{sec:emergence}

The geometric structure of the $Q = X \times S^1$ space provides a natural origin for the minimal coupling prescription of quantum mechanics. By considering a 5D field on this extended space and decomposing it into fiber harmonics, we can show that the resulting 4D effective theory naturally contains the covariant derivative, leading to the Klein-Gordon and, in the non-relativistic limit, the Schr\"odinger equation.

\subsection{A minimal action (EFT starting point)}
A field-theoretic presentation begins by specifying an action that couples matter to the connection. A minimal gauge-invariant choice (sufficient for an EFT-level discussion) is
\begin{equation}
S = \int d^4x \, \sqrt{-g} \left[ -\frac{1}{4}F_{\mu\nu}F^{\mu\nu} + (D_\mu\Phi)^\ast (D^\mu\Phi) - V(\Phi) + \frac{f^2}{2}\, (\partial_\mu\Theta + \kappa A_\mu)(\partial^\mu\Theta + \kappa A^\mu) \right],
\end{equation}
where $\Phi$ is a complex scalar matter field, $D_\mu\equiv \nabla_\mu-i\kappa A_\mu$, and $V(\Phi)$ encodes self-interactions and any symmetry-breaking scale. The field $\Theta(x)$ is a 4D St\"uckelberg field with decay constant $f$; the gauge-invariant combination $\partial_\mu\Theta+\kappa A_\mu$ provides explicit 4D dynamics and generates a vector mass $m_A=\kappa f$ in unitary gauge.

\subsection{The Fiber Metric and 5D Klein-Gordon Equation}

We begin by defining the metric on the total space $Q$. We extend the 4D spacetime metric $g_{\mu\nu}$ with a fiber component, where the connection $A_\mu$ is incorporated into the metric itself:
\begin{equation}
ds^2 = G_{AB}\,\mathrm{d}q^A\,\mathrm{d}q^B = g_{\mu\nu}\,\mathrm{d}x^\mu\,\mathrm{d}x^\nu + R^2\big(\mathrm{d}\theta + \kappa A_\mu\,\mathrm{d}x^\mu\big)^2
\end{equation}
Here, $R$ is the radius of the $S^1$ fiber. A free scalar field $\Phi(x, \theta)$ on this 5D space obeys the Klein-Gordon equation:
\begin{equation}
\frac{1}{\sqrt{-G}}\partial_A(\sqrt{-G}G^{AB}\partial_B\Phi) - M^2\Phi = 0
\end{equation}
where $M$ is the 5D mass of the field.

\subsection{Harmonic Expansion and Dimensional Reduction}

Since the fiber is a compact circle, we can expand $\Phi(x, \theta)$ in a Fourier series of fiber harmonics:
\begin{equation}
\Phi(x, \theta) = \sum_{n=-\infty}^{\infty} \Phi_n(x) e^{in\theta}
\end{equation}
where $\Phi_n(x)$ are the 4D field components for each mode $n$.

Substituting this expansion into the Klein-Gordon equation and integrating over $\theta$, we perform a dimensional reduction. The derivatives with respect to $\theta$ become algebraic terms involving $n$. The crucial step is the appearance of the covariant derivative. The 4D operator that acts on the $\phi_n(x)$ modes is precisely the minimally coupled d'Alembertian:
\begin{equation}
\left( D_\mu D^\mu + m_n^2 \right) \Phi_n(x) = 0
\end{equation}
where the covariant derivative is given by:
\begin{equation}
D_\mu = \nabla_\mu - i\,n\kappa\, A_\mu
\end{equation}
which is the standard KK result: schematically $\partial_\mu\to \partial_\mu+\kappa A_\mu\partial_\theta$, and for a mode $e^{in\theta}$ one has $\partial_\theta\to in$. Thus the effective 4D charge of the $n$-th mode is $q_n\equiv n\kappa$ (up to conventions for whether $\theta$ is dimensionless and how $R$ is absorbed). The KK mass tower is
\begin{equation}
m_n^2 = M^2 + \frac{n^2}{R^2}.
\end{equation}

\subsection{Non-relativistic Limit: The Schr\"odinger Equation}

In the non-relativistic limit, we consider the $n=1$ mode and write the field as $\Phi_1(x) = e^{-imt} \psi(x,t) / \sqrt{2m}$. Taking the low-energy limit of the Klein-Gordon equation, we recover the familiar Schr\"odinger equation for a particle of charge $q$ in an electromagnetic potential $A_\mu$:
\begin{equation}
i\hbar \frac{\partial\psi}{\partial t} = \left[ \frac{1}{2m}(-i\hbar\nabla - q\mathbf{A})^2 + qA_0 \right]\psi
\end{equation}
where $A_0$ is the scalar potential.

This derivation demonstrates that the gauge coupling and phase structure of quantum mechanics are not ad-hoc additions but emerge naturally from the geometry of the $S^1$ fiber bundle. It provides a geometric origin for the principle of minimal coupling.
